\section*{X86-64 Registers}

\begin{tabular} {l | l | l | l | l | l} \hline
  64B register & 32B & 16B & 8B & Description & Saver \\ \hline
  \tt{rax} & \tt{eax} & \tt{ax} & \tt{al} & Return Value & Caller \\
  \tt{rbx} & \tt{ebx} & \tt{bx} & \tt{bl} & Saved Register & Callee \\
  \tt{rcx} & \tt{ecx} & \tt{cx} & \tt{cl} & Arg 4 & Caller \\
  \tt{rdx} & \tt{edx} & \tt{dx} & \tt{dl} & Arg 3 & Caller \\
  \tt{rsi} & \tt{esi} & \tt{si} & \tt{sil} & Arg 2 & Caller \\
  \tt{rdi} & \tt{edi} & \tt{di} & \tt{dil} & Arg 1 & Caller \\
  \tt{rsp} & \tt{esp} & \tt{sp} & \tt{sil} & Stack Pointer & N/A \\
  \tt{rbp} & \tt{ebp} & \tt{bp} & \tt{bpl} & Frame Pointer & Callee \\
  \tt{r8} & \tt{r8d} & \tt{r8w} & \tt{r8b} & Arg 5 & Caller \\
  \tt{r9} & \tt{r9d} & \tt{r9w} & \tt{r9b} & Arg 6  & Caller \\
  \tt{r10} & \tt{r10d} & \tt{r10w} & \tt{r10b} & Scratch & Caller \\
  \tt{r11} & \tt{r11d} & \tt{r11w} & \tt{r11b} & Scratch & Caller \\
  \tt{r12} & \tt{r12d} & \tt{r12w} & \tt{r12b} & Saved Register & Callee \\
  \tt{r13} & \tt{r13d} & \tt{r13w} & \tt{r13b} & Saved Register & Callee \\
  \tt{r14} & \tt{r14d} & \tt{r14w} & \tt{r14b} & Saved Register & Callee \\
  \tt{r15} & \tt{r15d} & \tt{r15w} & \tt{r15b} & Saved Register & Callee \\
\end{tabular}

\vspace{0.2in} Most x86 instructions are a 2-argument form, with one
of the arguments capable of being a memory location, not just a
register or an immediate.

\vspace{0.2in}

\begin{tabular}{l|l|l} \hline
  Mode & Meaning & Examples \\ \hline
  Immediate & A 32b immediate value & \tt 0x381 \\
  & & 42 \\ \hline
  Register & A register  & \tt rax \\
  & & \tt r11b \\ \hline
  Memory Location & Memory location & \tt [rax] \\
  & & \tt [r11 + 32] \\ \hline
  Indexed Memory Location & The second register is multiplied by & \tt [rax + 8 * rbp] \\
  & the stride (1, 2, 4, or 8 only) & [r11 + 4 * r12 + 32] \\ \hline
  
  \end{tabular}

\vspace{0.2in} We use Intel syntax, which automatically inferrs the
bit width for operations based on the register involved.  For the rare
case where one wishes to use an instruction with just an immediate and
memory, we need to specify the bit width explicitly either by writing
the immediate into a register first or by annotating the memory
location with {\tt byte ptr}, {\tt word ptr}, {\tt dword ptr}, or {\tt
  qword ptr} for 1, 2, 4, and 8 bits respectively.

One can see the large variety possible through addressing in the following eamples for {\tt mov}.

\vspace{0.2in}

\begin{tabular}{l|l}
  Example & Meaning \\ \hline
  \tt mov rdi, rsi & Move the contents of register {\tt rsi} into {\tt rdi} \\
  \tt mov rdi, [rsi] & Load the memory pointed to by {\tt rsi} (equivalent to RISC-V {\tt lw} \\ 
  \tt mov rdi, 0xf00d & Load the immediate into {\tt rdi}.  \\
  \tt mov [rsi + 8], rdi & Store the contents of {\tt rdi} into the memory location {\tt rsi + 8} \\
  \tt mov [rsi + rdx * 8 + 32], rdi & Store the contents of {\tt rdi} into the memory location {\tt rsi + rdi * 8 + 32} \\
  \tt mov dword ptr [rsi + 32], -42 & Store the 32b immediate value -42 into the memory location {\tt rsi + 32} \\
  
  \end{tabular}

\clearpage

\section{x86-64 Instructions}

For our x86-64 programming we are using a deliberately RISC-like
subset of x86, specifically excluding instructions like {\tt push} and
{\tt pop} that manipulate the stack.

\vspace{0.2in}
\begin{tabular}{l|l} \hline
  Instruction & Meaning \\ \hline
\tt add dst, src & Add the source and destination together, storing the
result in the destination \\
\tt and dst, src & Performs a logical AND between the source and
destination, storing the results in the destination \\
\tt call target & Control flow transfer to label or register, decrements rsp by 8,
storing rip into [rsp] \\
\tt cmov{\it cc} dst, src & Conditional move from source to dst. See condition
codes \\
\tt cmp src1, src2 & Compare the two values and set the flags for conditional
move and branch \\
\tt div arg & Unsigned division: This takes a 128b number composed
of {\tt rdx:rax} (that is, the upper 64b are in \\
  & {\tt rdx}) and uses
the arg as the divisor, treating the arguments as signed.
The quotient is placed in {\tt rax} \\
& while the remainder is in {\tt rdx} \\
\tt idiv arg & Signed division: This takes a 128b number composed
of {\tt rdx:rax} (that is, the upper 64b are in \\
  & {\tt rdx}) and uses
the arg as the divisor, treating the arguments as signed.
The quotient is placed in {\tt rax} \\
& while the remainder is in {\tt rdx} \\
\tt imul dst, src & Unsigned multiplication: This multiplies the source and
destination together, \\ & writing it in the destination. The
destination must be a register. \\ 
\tt imul dst, src, imm & Unsigned multiplication: This multiplies the source by
the 32b sign-extended immediate, \\ & writing it in the
destination. The destination must be a register. \\
\tt j{\it cc} dst & Conditional jump: Jump to the destination if the condition
is true. See condition codes \\ 
\tt jmp dst & Transfer control flow to the destination \\
\tt lea dst, src & Load Effective Immediate: Load the address referred to
in src into dst. \\
\tt mul src & Signed multiplication: This multiples the source by the
contents of the {\tt rax} register, \\ & storing the lower 64b of the
results in {\tt rax} and the upper 64b in {\tt rdx}. \\
\tt mov dst, src & Move src into destination \\
\tt neg dst & 2s complement negation of the destination \\
\tt not dst & Bitwise negation of the destination \\
\tt or dst, src & Performs a logical OR between the source and
destination, writing the results in the destination \\
\tt ret & Loads rip as [rsp], increments rsp by 8, transfers
control flow to rip \\ 
\tt shl dst, src & Shifts the destination to the left by src bits \\
\tt sar dst, src & Shifts the destination to the right by src bits, shifting in
the sign bit \\ 
\tt shr dst, src & Shifts the destination to the right by src bits, shifting in 0
bits on the left \\
\tt sub dst, src & Calculates dst - src, writing the results into dst \\
\tt xor dst, src & Performs a logical XOR between the source and
destination, writing the results in the destination \\ \hline
\end{tabular}

\vspace{.2in} X86 uses a ``compare and conditional operation'' motif
for branches and conditional moves.  Many instructions set various
condition flags, but the most common is {\tt cmp} which performs a
subtraction but only sets the flags.  This is not all of the mnemonics
but it is the most important set.

\vspace{0.2in}
\begin{tabular}{l|l|l}
  Code & Meaning & Examples \\
\tt a & Above (unsigned $>$) & \tt cmova, ja \\
\tt ae & Above or equal (unsigned $>=$) & \tt cmovae, jae  \\
\tt b & Below (unsigned $<$) & \tt cmovb, jb \\ 
\tt be & Below or equal (unsigned <=) & \tt  cmovbe, jbe \\ 
\tt e & Equal  & \tt cmove, je \\
\tt g &   Greater (signed >)  & \tt  cmovg, jg \\
\tt  ge & Greater or equal (signed >=)  & \tt  cmovge, jge \\
\tt l & Less than (signed <)  & \tt  cmovl, jl  \\
\tt le & Less than or equal (signed <=)  & \tt  cmovle, jle  \\
\tt ne & Not equal  & \tt  cmovne, jne  \\
\tt ns & Not sign (result was positive)  & \tt  cmovns, jns \\
\tt nz & Not zero  & \tt  cmovnz, jnz  \\
\tt s & Sign (result was negative)  & \tt  cmovs, js \\
\tt z & Zero (result was zero) & \tt  cmovz, jz \\
\end{tabular}


