\section*{Control and Status Registers}
\small
\begin{tabular}{l|l|l}
Instruction & Operation & Notes \\ \hline
{\tt csrrw rd csr rs} & {\tt csr = rs; rd = csr} & {\tt rd=x0} only does the write \\
{\tt csrrs rd csr rs} & {\tt csr = rs | csr; rd = csr} & {\tt rs=x0} only does the read \\
{\tt csrrc rd csr rs} & {\tt csr = \^rs \& csr; rd = csr} & {\tt rs=x0} only does the read \\
{\tt csrrwi rd csr imm} & {\tt csr = imm, rd = csr} & {\tt rd=x0} only writes.  Immediate is 5 bit, zero extended \\
{\tt csrrsi rd csr imm} & {\tt csr = imm | csr, rd = csr} & {\tt imm=0} only reads.  Immediate is 5 bit, zero extended \\
{\tt csrrci rd csr imm} & {\tt csr = \^imm \& csr, rd = csr} & {\tt imm=0} only reads.  Immediate is 5 bit, zero extended \\
\end{tabular}

\vspace{0.2in}
Pseudo-instructions {\tt csrw} and {\tt csrr} are converted to {\tt
  csrrw} and {\tt csrrs} respectively.  All CSR instructions are
I-type with the immediate specifying the CSR.  The immediate CSRs use
the source register as a 5-bit, zero-extended immediate.  Traps are
returned from using {\tt mret}.

The listing of CSRs is deliberately incomplete but it
is designed for both a ECS 50 (assembly) class and an ECS 150
(operating systems) type class as a quick reference.
\vspace{0.2in}

\begin{tabular}{l|l}
  CSR & Meaning and Access \\ \hline
  \tt misa & Machine ISA: Allows checking what extensions are supported and disabling them. \\
  \tt mvendorid & Manufacturer ID number \\
  \tt marchid & Architeture ID number \\
  \tt mhartid & Hart (HARdware Thread) number \\
  \tt mstatus & The lower 32b of machine status. \\
  \tt mstatush & The upper 32b of machine status. \\
  \tt mtvec & Address of the trap handler.  If the last bit is set the interrupt portion is vectored. \\
  {\tt medeleg} and {\tt mideleg} & Delegation of interrupts \\
  {\tt mip} and {\tt mie} & Pending interrupts and enabled interrupts \\
  \tt mepc & Machine Exception Program Counter.  The PC that triggered a trap. \\
  \tt mcause & Machine Cause.  The cause of the current trap/interrupt. \\
  \tt mtval & Additional value information.  For memory faults, the address fetched that triggered the fault. \\

  \tt pmpcfg0-pmpcfg15 & Physical Memory Protection configuration \\
  \tt pmpaddr0-pmpaddr63 & Physical Memory Protection Addresses \\
  \tt sapt & System Page Table pointer \\
  
\end{tabular}

\vspace{0.2in}

Comments on particular CSRs

{\tt mhartid}: RISC-V specifies a ``Hart'' as a Hardware thread of
execution.  In the absence of symmetric multithreading, this is the
CPU core number.  If there is hardware multithreading it specifies the
distinct hardware thread where multiple threads can share a single CPU
core.

{\tt mstatus} and {\tt mstatush}: These CSRs ({\tt mstatush} only
exists on RV32, RV64 just has a combined {\tt mstatus}) state both the
current state of the machine in terms of operating mode and other
operating data.  This includes whether interrupts are enabled, what
mode the system is currently in, and what mode the system will be
returned to with {\tt mret} or {\tt sret} when returning from the trap
handler.  This CSR is very complicated with lots of details for those
implementing operating systems or similar low level code.

{\tt mtvec}: This is the address of the trap handler.  When an
interrupt or trap occurs the CPU will update the operating mode and
jump to the trap handler code, disabling interrupts so the trap
handler won't be interrupted during the process.  If the last two bits
are {\tt 01b} the interrupts are vectored: jumping to {\tt mtvec +
  4*{\it Interrupt number}}.

{\tt medeleg} and {\tt mideleg}: RISC-V supports up to 32 distinct
traps and interrupts.  This register allows machine mode (M-mode) to
delegate specific traps and interrupts to the S-mode, so that those
traps are instead handled by the S-mode trap handler.  Only traps that
occur in S-mode or U-mode are delegated, any trap or interrupt that
occurs while in M-mode will never be delegated to the S-mode trap
handler.

{\tt mscratch}: This register generally holds a pointer to space for
the trap handler.  The common convention for the trap handler is to
perform an atomic swap with {\tt sp}, giving the trap handler its own
stack space.  The trap handler can then save all registers.  Once this
is complete the trap handler can now operate normally.  When exiting
the trap handler the trap handler should restore all the registers,
restore {\tt sp} by again swapping it with {\tt mscratch}, and then
execute {\tt mret}.

{\tt mepc}: The address of the instruction triggering a trap.  For an
exception this is the instruction which triggered the exception (and
is not yet executed) and for an interrupt it is the first instruction
not yet executed.  If the trap handler wants to retry the instruction
(e.g. because it was an interrupt or because it was a corrected memory
protection error), {\tt mret} will just return to the address in {\tt
  mepc}.  If the trap handler instead corrected the exception (e.g. a
misaligned load), it should first increment {\tt mepc} to the next
instruction before returning.

{\tt mcause}: The particular trap/interrupt number that triggered the
exception.  Interrupts have the MSB set (allowing an easy signed
comparison with 0).

{\tt mtval}: The value depends on the particular trap or interrupt.
For memory related faults (e.g. misaligned load, page fault), it
always contains the virtual address of the memory who's access
triggered the fault.

{\tt pmpcfg*} and {\tt pmpaddr*}: These CSRs are to implement a low
level Physical Memory Protection scheme, allowing M-mode to control
what memory is accessed in S-Mode or U-Mode.  Its primary usage is on
small embedded processor that don't run a full virtual-memory
supporting operating system but still want some level of protection.

The config registers use 8 bits to indicate the status of each
corresponding address register.  These include whether S/U mode can
Read data, Write data, eXecute data, the Address mode for the address
register, and whether to Lock out control.

The two primary address modes are to specify a Naturally Aligned Power
of 2 (NAPOT) in the corresponding address register or whether two
addresses should be taken together to specify a valid range.

{\tt satp}: This is the pointer to the page table.  For efficiency
reasons it points to a physical page rather than a specific memory
location, because page table entries are aligned on page-boundaries.
It only exists if the processor supports S mode, and it can be applied
for both S-mode and U-mode accesses.

The page table itself is a 2-level radix structure: Pages in RISC-V
are 4 kB which means the page offset a 12b value and the virtual
address is 20b.  Since pages are 4 kB, and page table entries are 4B,
this means that we can fit 1024 page table entries in a physical page.

The SATP pointer points to the root physical page: The upper 10 bits
of the virtual address are used to look up the individual entry.  If
the entry is invalid it triggers a trap.  If it is valid the resulting
entry specifies another page to evaluate the next 10 bits of the
vitrual address.  Once the proper page table entry is found the
physical address then replaces the virtual address before the actual
memory fetch occurs.

One particular note is that the S and U modes can use the same page
table.  In that case the system's pages are usually marked as both
``global'' for efficiently 23098and unaccessable to U mode.

